\documentclass[a4paper]{article}

\usepackage[paper=a4paper, left=1.5cm, right=1.5cm, bottom=1.5cm, top=3.5cm]{geometry}
\usepackage[spanish,activeacute]{babel}
\usepackage[utf8]{inputenc}
\usepackage{amsthm}
\usepackage{amsmath}
\usepackage{amsfonts}
\usepackage{amssymb}
\usepackage{alltt}
\usepackage{graphicx} %Para incluir el logo de la UBA
\usepackage{caratula} %Para armar el cuadro de integrantes
\usepackage{multirow} %Para poder poner varias lineas juntas sin divisiones en una tabla
\usepackage[lined,ruled,linesnumbered]{algorithm2e}
\usepackage{algpseudocode}
\usepackage{scrextend}
\usepackage{blindtext}
\usepackage{float}

\providecommand{\keywords}[1]{\textbf{\textit{Palabras Clave---}} #1}

%Cosas para escribir codigo fuente
%Fuente: http://en.wikibooks.org/wiki/LaTeX/Source_Code_Listings
\usepackage{listings}
\usepackage{color}

\setcounter{secnumdepth}{5}

\definecolor{mygreen}{rgb}{0,0.6,0}
\definecolor{mygray}{rgb}{0.5,0.5,0.5}
\definecolor{myorange}{rgb}{1,0.4,0.2}
\definecolor{myblue}{rgb}{0,0,0.65}

%Configuracion para los listings
\lstset{ %
	keepspaces=true,
  backgroundcolor=\color{white},   % choose the background color; you must add \usepackage{color} or \usepackage{xcolor}
  basicstyle=\small,        % the size of the fonts that are used for the code
  breakatwhitespace=false,         % sets if automatic breaks should only happen at whitespace
  breaklines=true,                 % sets automatic line breaking
  captionpos=b,                    % sets the caption-position to bottom
  commentstyle=\color{mygreen},    % comment style
  deletekeywords={...},            % if you want to delete keywords from the given language
  escapeinside={\%*}{*)},          % if you want to add LaTeX within your code
  extendedchars=true,              % lets you use non-ASCII characters; for 8-bits encodings only, does not work with UTF-8
  frame=none,                    % adds a frame around the code
  keywordstyle=\color{myblue},       % keyword style
  language=Octave,                 % the language of the code
  morekeywords={*,...},            % if you want to add more keywords to the set
  numbers=none,                    % where to put the line-numbers; possible values are (none, left, right)
  numbersep=5pt,                   % how far the line-numbers are from the code
  numberstyle=\tiny\color{mygray}, % the style that is used for the line-numbers
  rulecolor=\color{black},         % if not set, the frame-color may be changed on line-breaks within not-black text (e.g. comments (green here))
  showspaces=false,                % show spaces everywhere adding particular underscores; it overrides 'showstringspaces'
  showstringspaces=false,          % underline spaces within strings only
  showtabs=false,                  % show tabs within strings adding particular underscores
  stepnumber=1,                    % the step between two line-numbers. If it's 1, each line will be numbered
  stringstyle=\color{myorange},     % string literal style
  tabsize=2,                       % sets default tabsize to 2 spaces
  title=\lstname                   % show the filename of files included with \lstinputlisting; also try caption instead of title
}

\renewcommand{\lstlistingname}{C\'{o}digo}
\newcommand{\real}{\mathbb{R}}
\lstset{language=Prolog}


%\topmargin = -1cm
%\textheight = 24cm 

\begin{document}

\integrante{Aisemberg, Dan}{242/12}{dea4493@hotmail.com}
\integrante{Almansi, Emilio}{674/12}{ealmansi@gmail.com}
\integrante{Levy Alfie, Jon\'as}{081/12}{jonaslevy5@gmail.com}

\def\Materia{Paradigmas de Lenguajes de Programación}
\def\Titulo{Trabajo Pr\'{a}ctico - Programación Lógica}
\def\Fecha{4 de noviembre de 2014}
\def\Grupo{Grupo: Resolución Súper Lógico Deteminística}

%----- CARATULA -----%

\thispagestyle{empty}

\begin{center}
	\includegraphics[scale = 0.25]{logo_uba.jpg}
\end{center}

\vspace{5mm}

\begin{center}
	{\textbf{\large UNIVERSIDAD DE BUENOS AIRES}}\\[1.5em]
	{\textbf{\large Departamento de Computaci\'{o}n}}\\[1.5em]
    {\textbf{\large Facultad de Ciencias Exactas y Naturales}}\\
    \vspace{35mm}
    {\LARGE\textbf{\Materia}}\\[1em]    
    \vspace{15mm}
    {\Large \textbf{\Titulo}}\\[1em]
    \vspace{15mm}
    {\textbf{\Large \Fecha}}\\
    \vspace{15mm}
    {\textbf{\Large \Grupo}}\\
    \vspace{15mm}
    \textbf{\tablaints}
\end{center}

\newtheorem{teo}{Teorema}[section]
\newtheorem{propo}{Proposici\'{o}n}[section]
\newtheorem{lema}{Lema}[section]
\newtheorem{coro}{Corolario}[section]
\newtheorem{defi}{Definici\'{o}n}[section]

\newpage
\thispagestyle{empty}
%\tableofcontents

\parskip=5pt
\setlength{\parindent}{0pt}

\newpage
\setcounter{page}{1}
\pagenumbering{arabic}
\pagestyle{plain}

\newpage


\newcommand{\Asig}{\ensuremath{\leftarrow}}
\newcommand{\AndY}{\ensuremath{\wedge}}
\newcommand{\Or}{\ensuremath{\vee}}
\newcommand{\Not}{\ensuremath{\neg}}
\newcommand{\NotEq}{\ensuremath{\neq}}
\newcommand{\MayorIg}{\ensuremath{\geq}}
\newcommand{\tabu}{\hspace*{0.7cm}}
\newcommand{\ctabu}{\hspace*{0.8cm}}
\newcommand{\htabu}{\hspace*{0.35cm}}
\newcommand{\moduloNombre}[1]{\textbf{#1}}

%\include{ej}

%
%\bibliographystyle{plain}
%\bibliography{tp3}
\lstinputlisting[language=Prolog, caption=Implementación y tests]{../tpl.pl}

\end{document}
